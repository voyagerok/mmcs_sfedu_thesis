\documentclass[10pt]{beamer}

\usepackage{amssymb,amsmath,mathtext}
\usepackage{indentfirst,amsfonts}
\usepackage{makecell,multirow,longtable}
\usepackage{verbatim}
\usepackage {graphicx}

\usepackage[russian]{babel}
\usepackage[T2A]{fontenc}
\usepackage[utf8]{inputenc}
\usepackage {paratype}

\PassOptionsToPackage{unicode}{hyperref}
\PassOptionsToPackage{naturalnames}{hyperref}

\setbeamertemplate{navigation symbols}{}

\usetheme{Madrid}
%\usetheme{Boadilla}
%\usetheme{Warsaw}


\beamersetuncovermixins{\opaqueness<1>{30}}{\opaqueness<2->{25}}

\deftranslation[to=russian]{example}{пример}
\deftranslation[to=russian]{Example}{Пример}

\makeatletter
\setbeamertemplate{footline}
{
  \leavevmode%
  \hbox{%
  \begin{beamercolorbox}[wd=.333333\paperwidth,ht=2.25ex,dp=1ex,center]{author in head/foot}%
  \end{beamercolorbox}%
  \begin{beamercolorbox}[wd=.333333\paperwidth,ht=2.25ex,dp=1ex,center]{title in head/foot}%
  \end{beamercolorbox}%
  \begin{beamercolorbox}[wd=.333333\paperwidth,ht=2.25ex,dp=1ex,right]{date in head/foot}%
    \usebeamerfont{date in head/foot}\insertshortdate{}\hspace*{2em}
    \insertframenumber{} / \inserttotalframenumber\hspace*{2ex} 
  \end{beamercolorbox}}%
  \vskip0pt%
}
\makeatother

\begin{document}
\title[Алгоритмы извлечения полей]{Алгоритмы и программная система извлечения информационных полей из слабоструктурированных документов}
\author{Н. С. Линецкий\\
\ \\
Направление подготовки:\\
Фундаментальная информатика и информационные технологии\\
\ \\
Руководитель:\\
доцент, к.т.н. М. Г. Адигеев}
\date{Ростов-на-Дону, 2017} 
\institute{Южный Федеральный Университет\\
Институт математики, механики и компьютерных наук имени И. И. Воровича}

{
	\setbeamertemplate{footline}{} 
	\begin{frame}
		\titlepage
	\end{frame}
}

\begin{frame}{Содержание}
\tableofcontents
\end{frame}

\section{Постановка задачи}
\begin{frame}
\frametitle{Постановка задачи}
\begin{itemize}
	\item Изучение существующих алгоритмов и программных решений для извлечения информации из текстов на естественном языке
	\item Разработка системы для извлечения реквизитов с учетом специфики слабоструктурированных документов
	\item Обеспечение возможности конфигурировать систему извне путем задания правил, описывающих шаблоны полей
\end{itemize}
\end{frame}

\section{Введение}
\begin{frame}
\frametitle{Введение}
\begin{itemize}
	\item Необходимость автоматизации процесса обработки электронных документов
	\item Существующие решения не учитывают специфику работы со слабоструктурированными текстами
\end{itemize}
\begin{example}
Гражданин Иванов Иван Иванович, именуемый в дальнейшем <<Наймодатель>>, с одной стороны, и гражданин Петров Петр Петрович, именуемый в дальнейшем <<Наниматель>>, с другой стороны, именуемые в дальнейшем <<Стороны>>, заключили настоящий договор, в дальнейшем <<Договор>>, о нижеследующем:
\end{example}
\end{frame}

\section{Концепция решения}
\begin{frame}
\frametitle{Концепция решения: структура}
Три основных модуля:
\begin{itemize}
	\item Разметка входного текста договора
	\item Анализ входных правил, описывающих шаблоны искомых полей
	\item Извлечение полей с использованием алгоритма Томиты
\end{itemize}
\end{frame}

\begin{frame}[fragile]
\frametitle{Концепция решения: алгоритм Томиты}
\begin{itemize}
	\item Разработан японским ученым Масару Томитой
	\item Generalized LR парсер
	\item Graph Structured Stack: стек разбора с поддержкой операций ветвления и слияния
\end{itemize}
\begin{figure}% p означает, что нужно выделить для рисунка
\centering
\includegraphics[width=0.7\textwidth]{img/gss-step5.png}
\end{figure}
\end{frame}

\begin{frame}
\frametitle{Концепция решения: Томита-парсер}
\begin{itemize}
	\item Программа для извлечени фактов из текстов на естественном языке
	\item Разработана в Яндексе, исходные коды открыты
	\item Реализует алгоритм GLR-парсинга
	\item Не имеет средств задания контекста для искомых данных
\end{itemize}
\end{frame}

\section{Реализация}
\begin{frame}[fragile]
\frametitle{Реализация: разметка текста}
\begin{itemize}
	\item Разложение текста на элементарные конструкции - слова
	\item Выделение морфологических и семантических свойств для слов
	\item Морфологический анализатор - pymorphy2:
	\begin{itemize}
		\item Аанализа на основе данных из OpenCorpora
		\item Реализована поддержка разбора несловарных слов
	\end{itemize}
\end{itemize}
\begin{example}[Разбор слова]
Результат разбора слова <<стали>> с помощью pymorphy2:
\begin{enumerate}
	\item Нормальная форма: стать, глагол, прошедшее время, совершенный вид, множественное число, индекс уверенности: 0.78
	\item Нормальная форма: сталь, существительное, женский род, единственное число, индекс уверенности: 0.22
\end{enumerate}
\end{example}
\end{frame}

\begin{frame}
\frametitle{Реализация: пользовательские правила}
\begin{itemize}
	\item Две секции: секция правил и секция команд
	\item Запись правил производится в формате $S = right\_handle_1\ |\ right\_handle_2\ |\ ...\ |\ right\_handle_n$, где $S$ - имя искомого реквизита, $right\_handle_i$ - шаблон, описывающий реквизит
	\item Шаблон реквизита может содержать как имена других реквизитов, так и произвольные слова
	\item Для элементов шаблона разрешено задавать дополнительные ограничения, например: длина, регистр, регулярное выражение
	\item Определен ряд встроенных правил и слов
\end{itemize}
\end{frame}

\begin{frame}[fragile]
\frametitle{Реализация: пользовательские правила}
\begin{example}[Запись встроенного правила, описывающего полное имя человека]
\begin{verbatim}
PersonFullName = Surname SecPart;
SecPart = FullRec | ShordRec;
FullRec = name patr;
ShortRec = init init;
Surname = surn | word(upper1);
\end{verbatim}
\end{example}
\end{frame}

\begin{frame}[fragile]
\frametitle{Реализация: пользовательские правила}
\begin{itemize}
	\item В секции команд указываются искомые правила c дополнительными параметрами поиска
	\item Команда Find:\\ 
	$Find\ rule\_name\ [hint\_words]\ [dependencies]\ [alias]$, где\\
	$rule\_name$ - имя искомого правила\\
	$dependencies$ - список правил-зависимостей\\
	$hint\_words$ - контекст\\
	$alias$ - псевдоним правила
\end{itemize}
\begin{example}
Find PersonFullName (hint\_words: "наниматель") as Employer;\\
Find FullDate with deps (left: Town) as AgreementDateValue;
\end{example}
\end{frame}

\begin{frame}
\frametitle{Реализация: извлечение полей}
\begin{itemize}
	\item Построение контекстно-свободной грамматики $G = (\Sigma, N, P, S$)
	\item Построение LR-анализатора
	\item Разбор входного текста с помощью алгоритма Томиты
	\begin{itemize}
		\item GSS формируется по одному уровню за итерацию
		\item Узлы текущего уровня - активные
	\end{itemize}
\end{itemize}
\begin{figure}% p означает, что нужно выделить для рисунка
\centering
\includegraphics[width=0.7\textwidth]{img/gss-step3.png}
\end{figure}
\end{frame}

\begin{frame}
\frametitle{Реализация: извлечение полей}
Этапы итерации:
\begin{description}
	\item[Actor] получение все $action$
	\item[Reducer] выполнение активных сверток
	\item[Shifter] выполнение активных сдвигов и формирование следующего уровня 
\end{description}
\begin{figure}% p означает, что нужно выделить для рисунка
\centering
\includegraphics[width=\textwidth]{img/glr-iteration.png}
\end{figure}
\end{frame}

\section{Результаты}
\begin{frame}
\frametitle{Результаты}
\begin{itemize}
	\item Изучены существующие алгоритмы и программные решения для извлечения информации из текстов на естественном языке
	\item Разработана система для извлечения реквизитов с учетом специфики слабоструктурированных документов
	\item Обеспечена возможность конфигурировать систему извне путем задания правил, описывающих шаблоны полей
\end{itemize}
\end{frame}

\end{document}